\documentclass[a4paper,11pt]{article}
\usepackage[utf8]{inputenc}    % Pour que LaTeX comprenne les accents.
\usepackage[T1]{fontenc}
\usepackage[french]{babel}     % Traitement du texte adapté aux règles typographiques
                               % de la langue donnée en option (e.g., pour l'espacement
                               % après les ponctuations
\usepackage{amsmath, amsthm, amssymb} 
\usepackage{graphicx}
\usepackage{xifthen}

%On définit des commandes pour pouvoir écrire dans la suite plus facilement les
%symboles très pénibles qu'on va utiliser très souvent. Genre les q-factorielles

% Usage: \eq{<name>}{<equation>}
\newcommand\eq[2]{
  \ifthenelse{\isempty{#1}}
             {\begin{equation*}#2\end{equation*}}
             {\begin{equation*}#2\tag{#1}\end{equation*}}
}

\title{Identités qui font pas bien dormir}
\author{Nous}


\begin{document}
\maketitle
La formule suivante est un découpage par rapport au carré de Durfee sur la série
génératrice des k partitions.
\eq{}{\sum \frac{z^nq^{n^2}}{(zq;q)(q;q)_n}=\frac{1}{(zq;q)_\infty}}

% Cauchy
\eq{Cauchy}{
  \sum \frac{(a;q)_n}{(q;q)_n}z^n=\frac{(az;q)_\infty}{(z;q)_\infty}
}

\eq{}{
  \sum \frac{1}{(q;q)_n}q^{\frac{n(n+1)}{2}}z^n=(-zq;q)_\infty
}

\eq{}{
  \sum \frac{1}{(q^2;q^2)_n}q^{n^2}=(-q;q^2)_\infty
}

% Heine
\eq{Heine}{
  \sum
  \frac{(a)_n(b)_n}{(q)_n(c)_n}z^n=
  \frac{(b)_\infty(az)_\infty}{(c)_\infty(z)_\infty}
  \sum \frac{(\frac{c}{b})_n(z)_n}{(q)_n(az)_n}b^n
}

% Gauss
\eq{Gauss}{
  \sum
  \frac{(a)_n(b)_n}{(q)_n(c)_n}(\frac{c}{ab})^n=
  \frac{(b)_\infty(\frac{c}{b})_\infty}{(c)_\infty(\frac{c}{ab})_\infty}
  \sum
  \frac{(\frac{c}{ab})_n}{(q)_n}b^n=
  \frac{(\frac{c}{a})_\infty(\frac{c}{b})_\infty}{(c)_\infty(\frac{c}{ab})_\infty}
}

\eq{}{
  (x)_n(-x)_n=(x^2;q^2)_n
}

% q-Kummer
\eq{q-Kummer}{
  \sum_{n \geq 0}
  \frac{(a)_n (b)_n (\frac{-q}{a})^n}{(q)_n (\frac{bq}{a})_n} = 
  \frac{(\frac{bq^2}{a^2};q^2)_\infty (bq;q^2)_\infty}{(q;q^2)_\infty (\frac{bq}{a})_\infty (\frac{-q}{a})_\infty}
}

% Lebesgue
Pour $a \rightarrow \infty$
\eq{Lebesgue}{
  \sum
  \frac{q^{{n+1} \choose 2}(b)_n}{(q)_n}=
  \frac{(bq;q^2)_\infty}{(q;q^2)_\infty}
}

%q-Pfaff-Saclshatz
Pour $|q| < 1$
\eq{q-Pfaff-Saclshatz}{
  \sum_{n=0}^N
  \frac{(a)_n (b)_n (q^{-N})_n q^n}{(c)_n(q)_n(\frac{abq^{1-N}}{c})_n} =
  \frac{(\frac{c}{a})_N (\frac{c}{b})_N}{(c)_N (\frac{c}{ab})_N}
}

Pour $b\rightarrow 0$
% q-vandermonde
\eq{q-Vandermonde}{
  \sum_{n=0}^N
  \frac{(a)_n (q^{-N})_n q^n}{(c)_n(q)_n} =
  \frac{(\frac{c}{a})_N}{(c)_N}a^N
}

% Binomial
\eq{}{
  \sum_{n=0}^{N}z^nq^{2 \choose k+1} {k \brack n} =(-zq)_n
}

% q-Gauss
\eq{q-Gauss}{
  \sum_{n \geq 0}
  \frac{(a)_n (b)_n (\frac{c}{ab})^n}{(c)_n(q)_n} =
  \frac{(\frac{c}{a})_\infty (\frac{c}{b})_\infty}{(c)_\infty (\frac{c}{ab})_\infty}
}

\end{document}

\documentclass[a4paper,11pt]{article}
\usepackage[utf8]{inputenc}    % Pour que LaTeX comprenne les accents.
\usepackage[T1]{fontenc}
\usepackage[french]{babel}     % Traitement du texte adapté aux règles typographiques
                               % de la langue donnée en option (e.g., pour l'espacement
                               % après les ponctuations
\usepackage{amsmath, amsthm, amssymb} 
\usepackage{graphicx} 

%On définit des commandes pour pouvoir écrire dans la suite plus facilement les
%symboles très pénibles qu'on va utiliser très souvent. Genre les q-factorielles

\title{Identités qui font pas bien dormir}
\author{Nous}


\begin{document}
\maketitle
La formule suivante est un découpage par rapport au carré de Durfee sur la série
génératrice des k partitions.
$$\sum \frac{z^nq^{n^2}}{(zq;q)(q;q)_n}=\frac{1}{(zq;q)_\infty}$$

%Cauchy
$$\sum \frac{(a;q)_n}{(q;q)_n}z^n=\frac{(az;q)_\infty}{(z;q)_\infty}$$
%Euler

$$\sum \frac{1}{(q;q)_n}q^{\frac{n(n+1)}{2}}z^n=(-zq;q)_\infty$$

$$\sum \frac{1}{(q^2;q^2)_n}q^{n^2}=(-q;q^2)_\infty$$

%Heine

$$\sum
\frac{(a)_n(b)_n}{(q)_n(c)_n}z^n=
\frac{(b)_\infty(az)_\infty}{(c)_\infty(z)_\infty}
\sum \frac{(\frac{c}{b})_n(z)_n}{(q)_n(az)n}b^n$$

%q-Kummer

%binomial


%q-gauss


\end{document}

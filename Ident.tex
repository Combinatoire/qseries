\documentclass[a4paper,11pt]{article}
\usepackage[utf8]{inputenc} % Pour que LaTeX comprenne les accents.
\usepackage[T1]{fontenc}
\usepackage[french]{babel} % Traitement du texte adapté aux règles
% de la langue donnée en option (e.g., pour l'espacement
% après les ponctuations
\usepackage{amsmath, amsthm, amssymb}
\usepackage{graphicx}
\usepackage{xifthen}
%On définit des commandes pour pouvoir écrire dans la suite plus facilement les
%symboles très pénibles qu'on va utiliser très souvent. Genre les q-factorielles
% Usage: \eq{<name>}{<equation>}
\newcommand\eq[2]{
\ifthenelse{\isempty{#1}}
{\begin{equation*}#2\end{equation*}}
{\begin{equation*}#2\tag{#1}\end{equation*}}
}
\title{Identités qui font pas bien dormir}
\author{Nous}
\begin{document}
\maketitle
La formule suivante est un découpage par rapport au carré de Durfee sur la série
génératrice des k partitions.
\eq{}{\sum \frac{z^nq^{n^2}}{(zq;q)(q;q)_n}=\frac{1}{(zq;q)_\infty}}
% Cauchy
\eq{Cauchy}{
\sum \frac{(a;q)_n}{(q;q)_n}z^n=\frac{(az;q)_\infty}{(z;q)_\infty}
}
\eq{}{
\sum \frac{1}{(q;q)_n}q^{\frac{n(n+1)}{2}}z^n=(-zq;q)_\infty
}
\eq{}{
\sum \frac{1}{(q^2;q^2)_n}q^{n^2}=(-q;q^2)_\infty
}
% Heine
\eq{Heine}{
\sum
\frac{(a)_n(b)_n}{(q)_n(c)_n}z^n=
\frac{(b)_\infty(az)_\infty}{(c)_\infty(z)_\infty}
\sum \frac{(\frac{c}{b})_n(z)_n}{(q)_n(az)_n}b^n
}
% Gauss
\eq{Gauss}{
\sum
\frac{(a)_n(b)_n}{(q)_n(c)_n}(\frac{c}{ab})^n=
\frac{(b)_\infty(\frac{c}{b})_\infty}{(c)_\infty(\frac{c}{ab})_\infty}
\sum
\frac{(\frac{c}{ab})_n}{(q)_n}b^n=
\frac{(\frac{c}{a})_\infty(\frac{c}{b})_\infty}{(c)_\infty(\frac{c}{ab})_\infty}
}
\eq{}{
(x)_n(-x)_n=(x^2;q^2)_n
}
% q-Kummer
\eq{q-Kummer}{
\sum_{n \geq 0}
\frac{(a)_n (b)_n (\frac{-q}{a})^n}{(q)_n (\frac{bq}{a})_n} =
\frac{(\frac{bq^2}{a^2};q^2)_\infty (bq;q^2)_\infty}{(q;q^2)_\infty
(\frac{bq}{a})_\infty (\frac{-q}{a})_\infty}
}
% Lebesgue
Pour $a \rightarrow \infty$
\eq{Lebesgue}{
\sum
\frac{q^{{n+1} \choose {2}}(b)_n}{(q)_n}=
\frac{(bq;q^2)_\infty}{(q;q^2)_\infty}
}
%q-Pfaff-Saclshatz
Pour $|q| < 1$
\eq{q-Pfaff-Saclshatz}{
\sum_{n=0}^N
\frac{(a)_n (b)_n (q^{-N})_n q^n}{(c)_n(q)_n(\frac{abq^{1-N}}{c})_n} =
\frac{(\frac{c}{a})_N (\frac{c}{b})_N}{(c)_N (\frac{c}{ab})_N}
}
Pour $b\rightarrow 0$
% q-vandermonde
\eq{q-Vandermonde}{
\sum_{n=0}^N
\frac{(a)_n (q^{-N})_n q^n}{(c)_n(q)_n} =
\frac{(\frac{c}{a})_N}{(c)_N}a^N
}
% Binomial
\eq{}{
\sum_{n=0}^{N}z^nq^{{k+1} \choose {2}} {n \brack k} =(-zq)_n
}
% q-Gauss
\eq{q-Gauss}{
\sum_{n \geq 0}
\frac{(a)_n (b)_n (\frac{c}{ab})^n}{(c)_n(q)_n} =
\frac{(\frac{c}{a})_\infty (\frac{c}{b})_\infty}{(c)_\infty
(\frac{c}{ab})_\infty}
}

%Jacobi
\eq{Jacobi}{\sum_{n\in\mathbb{Z}} z^nq^{{n+1}\choose 2} =
(\frac{1}{z})_\infty(-zq)_\infty(q)_\infty}
\eq{Corollaire Jacobi}{\sum_{n\in\mathbb{Z}} (-1)^nq^{\frac{n(3n+1)}{2}} =
(q)_\infty}

\eq{Corollaire Jacobi}{\sum_{n\in\mathbb{Z}} q^{{n+1} \choose 2} =
\frac{(q^2;q^2)_\infty}{(q;q^2)_\infty}}

\eq{Corollaire Jacobi}{\sum_{n\in\mathbb{Z}} (-1)^nq^{2} =
\frac{(q)_\infty}{(-q)_\infty}}

%Propriétés modulo
\eq{}{p(7n + 5) = 0 \pmod 7}

\eq{}{p(11n + 6) = 0 \pmod {11}}

\eq{}{\beta_k = k 4^{-1} \pmod {5^k} \Rightarrow p(5^k n + \beta_k) = 0 \pmod {5^k}}

\eq{}{M(r,5,5n+4) = \frac{1}{5} p(5n + 4)}

\eq{}{M(r,7,7n+5) = \frac{1}{7} p(7n + 5)}

\eq{}{M(r,11,11n+6) = \frac{1}{11} p(11n + 6)}

\subsection{Petits choses potentiellement utiles}

Le nombre de partitions en $n$ parts impaires est égal au nombre de partitions en parts distinctes. 
Bijection en fusionnant tant que possible les parts de même taille.


Le nombre de partitions de $n$ en parts non-divisibles par $k$ est égal au nombre de partitions de $n$ où chaque part apparaît au plus $k-1$ fois. 
Bijection en regroupant les parts qui apparaissent $\ge k$ fois.


Le nombre de partitions de $n$ n'ayant rien en dessous du carré de Durfee est égal au nombre de partitions de $n$ en parts qui diffèrent d'au moins 2.
Bijection en cassant le carré de Durfee en 2 selon la diagonale et en recollant dans l'autre sens.

Le \textbf{rang} d'une partition est égal à la taille de la plus grande part moins le nombre de parts.

Le \textbf{crank} d'une part est la taille de la plus grande part s'il n'y a pas de $1$, sinon le nombre de parts plus grandes
que le nombre de $1$ moins le nombre de $1$.

On note $M(r,m,n)$ le nombre de partitions de $n$ avec crank égal à $r \pmod m$.

\end{document}
